\documentclass[a4paper,11pt]{article}
\usepackage[T2A]{fontenc}
\usepackage[utf8]{inputenc}
\usepackage[english,russian]{babel}

\title{Практическая работа N1}
\author{студент Пономарев Василий Васильевич}
\begin{document}
\maketitle
\begin{center}
Перевод из десятеричной в двоичную систему и обратно.\\
мое число 57\\
перевод в двоичную систему
\end{center}
\[ 
\left[ \frac{57}{2} \right] mod2=1;
\]
\[ 
\left[ \frac{28}{2} \right] mod2=0;
\]
\[ 
\left[ \frac{14}{2} \right] mod2=0;
\]
\[ 
\left[ \frac{7}{2} \right] mod2=1;
\]
\[ 
\left[ \frac{3}{2} \right] mod2=1;
\]
\[ 
\left[ \frac{1}{2} \right] mod2=1.
\]
\\
\[
57_{10} = 111001_{2}
\]
\begin{center}
перевод в десятеричную систему
\end{center}
\[
111001_2=1*2^0+0*2^1+0*2^2+1*2^3+1*2^4+1*2^5=1+8+16+32=57
\]
\end{document}